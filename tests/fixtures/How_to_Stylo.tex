\documentclass[12pt,french,letterpaper]{article}
  \usepackage[utf8]{inputenc}
\usepackage[spanish,portuguese,german,ukrainian,english,french,italian,main=french]{babel}

\usepackage{grffile}
\graphicspath{{graphicspath}}
\usepackage{float}
\let\origfigure=\figure
\let\endorigfigure=\endfigure
\renewenvironment{figure}[1][]{%
  \origfigure[H]
}{%
  \endorigfigure
}
\usepackage{epigraph}
\usepackage{fancyhdr}
\pagestyle{fancy}
\usepackage[all]{nowidow}
\lhead{}
\chead{Stylo}
\rhead{}
\lfoot{}
\cfoot{\thepage}
\rfoot{}

\newlength{\cslhangindent}
\setlength{\cslhangindent}{1.5em}
\newlength{\csllabelwidth}
\setlength{\csllabelwidth}{3em}
\newlength{\cslentryspacingunit} % times entry-spacing
\setlength{\cslentryspacingunit}{\parskip}
\newenvironment{CSLReferences}[2] % #1 hanging-ident, #2 entry spacing
 {% don't indent paragraphs
  \setlength{\parindent}{0pt}
  % turn on hanging indent if param 1 is 1
  \ifodd #1
  \let\oldpar\par
  \def\par{\hangindent=\cslhangindent\oldpar}
  \fi
  % set entry spacing
  \setlength{\parskip}{#2\cslentryspacingunit}
 }%
 {}
\usepackage{calc}
\newcommand{\CSLBlock}[1]{#1\hfill\break}
\newcommand{\CSLLeftMargin}[1]{\parbox[t]{\csllabelwidth}{#1}}
\newcommand{\CSLRightInline}[1]{\parbox[t]{\linewidth - \csllabelwidth}{#1}\break}
\newcommand{\CSLIndent}[1]{\hspace{\cslhangindent}#1}
\newcommand{\frkwlang}{\selectlanguage{french}}
\newcommand{\enkwlang}{\selectlanguage{english}}
\newcommand{\eskwlang}{\selectlanguage{spanish}}
\newcommand{\ptkwlang}{\selectlanguage{portuguese}}
\newcommand{\dekwlang}{\selectlanguage{german}}
\newcommand{\ukkwlang}{\selectlanguage{ukrainian}}
\newcommand{\itkwlang}{\selectlanguage{italian}}
\newcommand{\arkwlang}{\selectlanguage{arabic}}

\usepackage{lmodern}
\usepackage{amssymb,amsmath}
\usepackage{ifxetex,ifluatex}
\usepackage{fixltx2e} % provides \textsubscript
\ifnum 0\ifxetex 1\fi\ifluatex 1\fi=0 % if pdftex
  \usepackage[T1]{fontenc}

\else % if luatex or xelatex

  \usepackage{unicode-math}


\newcommand{\titlekwfr}{
\textbf{Mot-clés}~:
}
\newcommand{\titlekwpt}{
\textbf{Palavras-chave}:
}
\newcommand{\titlekwen}{
\textbf{Keywords}:
}
\newcommand{\titlekwit}{
\textbf{Parole chiave}:
}
\newcommand{\titlekwes}{
\textbf{Palabras clave}:
}
\newcommand{\titlekwde}{
\textbf{Schlüsselwörter}:
}
\newcommand{\titlekwuk}{
\textbf{ключові слова}:
}


\newenvironment{keyword}{\noindent}{\\}


  \defaultfontfeatures{Ligatures=TeX,Scale=MatchLowercase}
\fi
% use upquote if available, for straight quotes in verbatim environments
\IfFileExists{upquote.sty}{\usepackage{upquote}}{}
% use microtype if available
\IfFileExists{microtype.sty}{%
\usepackage[]{microtype}
\UseMicrotypeSet[protrusion]{basicmath} % disable protrusion for tt fonts
}{}
\PassOptionsToPackage{hyphens}{url} % url is loaded by hyperref
\usepackage[unicode=true]{hyperref}
\hypersetup{
            pdftitle={Stylo : Un article type},
            pdfauthor={Marcello Vitali-Rosati, Nicolas Sauret},
            pdfkeywords={édition, bac-à-sable, publishing, sandbox},
            pdfborder={0 0 0},
            breaklinks=true}
\urlstyle{same}  % don't use monospace font for urls

\usepackage{color}
\usepackage{fancyvrb}
\newcommand{\VerbBar}{|}
\newcommand{\VERB}{\Verb[commandchars=\\\{\}]}
\DefineVerbatimEnvironment{Highlighting}{Verbatim}{commandchars=\\\{\}}
% Add ',fontsize=\small' for more characters per line
\newenvironment{Shaded}{}{}
\newcommand{\AlertTok}[1]{\textcolor[rgb]{1.00,0.00,0.00}{\textbf{#1}}}
\newcommand{\AnnotationTok}[1]{\textcolor[rgb]{0.38,0.63,0.69}{\textbf{\textit{#1}}}}
\newcommand{\AttributeTok}[1]{\textcolor[rgb]{0.49,0.56,0.16}{#1}}
\newcommand{\BaseNTok}[1]{\textcolor[rgb]{0.25,0.63,0.44}{#1}}
\newcommand{\BuiltInTok}[1]{\textcolor[rgb]{0.00,0.50,0.00}{#1}}
\newcommand{\CharTok}[1]{\textcolor[rgb]{0.25,0.44,0.63}{#1}}
\newcommand{\CommentTok}[1]{\textcolor[rgb]{0.38,0.63,0.69}{\textit{#1}}}
\newcommand{\CommentVarTok}[1]{\textcolor[rgb]{0.38,0.63,0.69}{\textbf{\textit{#1}}}}
\newcommand{\ConstantTok}[1]{\textcolor[rgb]{0.53,0.00,0.00}{#1}}
\newcommand{\ControlFlowTok}[1]{\textcolor[rgb]{0.00,0.44,0.13}{\textbf{#1}}}
\newcommand{\DataTypeTok}[1]{\textcolor[rgb]{0.56,0.13,0.00}{#1}}
\newcommand{\DecValTok}[1]{\textcolor[rgb]{0.25,0.63,0.44}{#1}}
\newcommand{\DocumentationTok}[1]{\textcolor[rgb]{0.73,0.13,0.13}{\textit{#1}}}
\newcommand{\ErrorTok}[1]{\textcolor[rgb]{1.00,0.00,0.00}{\textbf{#1}}}
\newcommand{\ExtensionTok}[1]{#1}
\newcommand{\FloatTok}[1]{\textcolor[rgb]{0.25,0.63,0.44}{#1}}
\newcommand{\FunctionTok}[1]{\textcolor[rgb]{0.02,0.16,0.49}{#1}}
\newcommand{\ImportTok}[1]{\textcolor[rgb]{0.00,0.50,0.00}{\textbf{#1}}}
\newcommand{\InformationTok}[1]{\textcolor[rgb]{0.38,0.63,0.69}{\textbf{\textit{#1}}}}
\newcommand{\KeywordTok}[1]{\textcolor[rgb]{0.00,0.44,0.13}{\textbf{#1}}}
\newcommand{\NormalTok}[1]{#1}
\newcommand{\OperatorTok}[1]{\textcolor[rgb]{0.40,0.40,0.40}{#1}}
\newcommand{\OtherTok}[1]{\textcolor[rgb]{0.00,0.44,0.13}{#1}}
\newcommand{\PreprocessorTok}[1]{\textcolor[rgb]{0.74,0.48,0.00}{#1}}
\newcommand{\RegionMarkerTok}[1]{#1}
\newcommand{\SpecialCharTok}[1]{\textcolor[rgb]{0.25,0.44,0.63}{#1}}
\newcommand{\SpecialStringTok}[1]{\textcolor[rgb]{0.73,0.40,0.53}{#1}}
\newcommand{\StringTok}[1]{\textcolor[rgb]{0.25,0.44,0.63}{#1}}
\newcommand{\VariableTok}[1]{\textcolor[rgb]{0.10,0.09,0.49}{#1}}
\newcommand{\VerbatimStringTok}[1]{\textcolor[rgb]{0.25,0.44,0.63}{#1}}
\newcommand{\WarningTok}[1]{\textcolor[rgb]{0.38,0.63,0.69}{\textbf{\textit{#1}}}}

\usepackage{graphicx,grffile}
\makeatletter
\def\maxwidth{\ifdim\Gin@nat@width>\linewidth\linewidth\else\Gin@nat@width\fi}
\def\maxheight{\ifdim\Gin@nat@height>\textheight\textheight\else\Gin@nat@height\fi}
\makeatother
% Scale images if necessary, so that they will not overflow the page
% margins by default, and it is still possible to overwrite the defaults
% using explicit options in \includegraphics[width, height, ...]{}
\setkeys{Gin}{width=\maxwidth,height=\maxheight,keepaspectratio}

\IfFileExists{parskip.sty}{%
\usepackage{parskip}
}{% else
\setlength{\parindent}{0pt}
\setlength{\parskip}{6pt plus 2pt minus 1pt}
}
\setlength{\emergencystretch}{3em}  % prevent overfull lines
\providecommand{\tightlist}{%
  \setlength{\itemsep}{0pt}\setlength{\parskip}{0pt}}
\setcounter{secnumdepth}{0}
% Redefines (sub)paragraphs to behave more like sections
\ifx\paragraph\undefined\else
\let\oldparagraph\paragraph
\renewcommand{\paragraph}[1]{\oldparagraph{#1}\mbox{}}
\fi
\ifx\subparagraph\undefined\else
\let\oldsubparagraph\subparagraph
\renewcommand{\subparagraph}[1]{\oldsubparagraph{#1}\mbox{}}
\fi

% set default figure placement to htbp
\makeatletter
\def\fps@figure{htbp}
\makeatother


\title{Stylo}
\providecommand{\subtitle}[1]{}
\subtitle{Un article type}
\author{Marcello Vitali-Rosati \and Nicolas Sauret}
\date{}

\begin{document}
\frkwlang
\maketitle
\frkwlang

\begin{abstract}

C'est article est un exemple d'article type édité sur \emph{Stylo}.
\emph{Stylo} est un éditeur d'article scientifique dédié aux sciences
humaines. Vous pouvez éditer cet article pour vous entraîner. Une
documentation plus complète est accessible en cliquant sur le lien
documentation.
\end{abstract}



\thispagestyle{empty}
\frkwlang
\begin{keyword}
\titlekwfr
	édition, bac-à-sable
\end{keyword}
\enkwlang
\begin{keyword}
\titlekwen
	publishing, sandbox
\end{keyword}

\frkwlang

\frkwlang
\hypertarget{introduction}{%
\subsection{Introduction}\label{introduction}}

Stylo est un éditeur de texte scientifique. Pour faire vos premiers pas
sur Stylo, commencez par éditer cet article.

Stylo utilise le format \emph{markdown} pour baliser et styler le texte.
Cet article présente la syntaxe de base du markdown, et une
documentation plus complète
\href{https://github.com/adam-p/markdown-here/wiki/Markdown-Cheatsheet}{est
accessible ici}.

Vous pouvez visualiser l'article à tout moment en cliquant sur le bouton
\textbf{Preview} dans le menu de gauche.

\hypertarget{les-titres}{%
\subsection{Les titres}\label{les-titres}}

Les titres de niveaux 2 doivent être balisés avec 2 \texttt{\#}
(\texttt{\#\#}) et non un seul, car le titre de niveau 1 correspond au
titre de l'article, déclaré dans les métadonnées.

\hypertarget{titres-de-niveau-3}{%
\subsubsection{Titres de niveau 3}\label{titres-de-niveau-3}}

Les titres de niveaux 3 doivent être balisés avec 3 \texttt{\#} et ainsi
de suite.

Un saut de ligne correspond au début d'un nouveau paragraphe.

\hypertarget{syntaxe-minimale}{%
\subsection{Syntaxe minimale}\label{syntaxe-minimale}}

\hypertarget{gras-et-italique}{%
\subsubsection{Gras et italique}\label{gras-et-italique}}

Voici du texte en \emph{italique}. Voici du texte en \textbf{gras}.

\hypertarget{commentaire}{%
\subsubsection{Commentaire}\label{commentaire}}

La ligne ci-dessous n'apparaitra pas dans le document final.

\hypertarget{images}{%
\subsubsection{Images}\label{images}}

On peut insérer des images:

\begin{figure}
\centering
\includegraphics{images/test-david-7dc926ed4014a96a908e5fb21de52329.png}
\caption{Titre de mon image}
\end{figure}

Notez que le «Titre de mon image» sera pris en compte comme légende de
l'image dans l'article.

\hypertarget{listes}{%
\subsubsection{Listes}\label{listes}}

Les listes non numérotées:

\begin{itemize}
\tightlist
\item
  item
\item
  item
\item
  item
\end{itemize}

Les listes numérotées:

\begin{enumerate}
\def\labelenumi{\arabic{enumi}.}
\tightlist
\item
  item
\item
  item
\item
  item
\end{enumerate}

L'ordre des chiffres n'est pas important:

\begin{enumerate}
\def\labelenumi{\arabic{enumi}.}
\tightlist
\item
  item
\item
  item
\item
  item
\item
  item
\end{enumerate}

Cette liste sera automatiquement ordonnée de 1 à 4.

\hypertarget{appareil-critique}{%
\subsection{Appareil critique}\label{appareil-critique}}

\hypertarget{notes-de-bas-de-page}{%
\subsubsection{Notes de bas de page}\label{notes-de-bas-de-page}}

Un appel de note de bas de page se fait ainsi\footnote{La note se trouve
  ensuite à la fin du texte.}. Par ailleurs, la note peut être déclarée
n'importe où dans le document\footnote{Voici une note déclarée en fin de
  document}, en fin de document ou juste en dessous par
exemple\footnote{Une note déclarée ``n'importe où'', ici, juste en
  dessous du paragraphe correspondant.}.

Le label de la note peut être ce que vous voulez : il peut être
indifféremment un chiffre ou une suite de caractères\footnote{Voici une
  note avec un label textuel.}.

Une note de bas de page peut aussi être écrite dans le corps du texte,
en sortant l'accent circonflexe des crochets\footnote{Ceci est une note
  de bas de page inline. Elle peut être aussi longue que vous voulez,
  elle sera transformée comme les autres en note de bas de page}.

\hypertarget{les-ruxe9fuxe9rences}{%
\subsubsection{Les références}\label{les-ruxe9fuxe9rences}}

Un article scientifique utilise des références. Vous pouvez soit
importer un fichier
\href{http://www.bib.umontreal.ca/lgb/BibTeX/default.htm}{bibtex} généré
par votre logiciel de gestion bibliographique (conseillé), ou bien créer
manuellement les références au format bibtex.

Les références sont ensuite insérées dans le texte grâce à leur
\emph{clé bibtex}. Pour récupérer la clé bibtex d'une référence, il
suffit de cliquer sur la référence souhaitée dans la liste des référence
ci-contre. La clé est alors ajouté à votre presse-papier, il suffit
ensuite de la coller dans le texte
(\protect\hyperlink{ref-goody_raison_1979}{Goody 1979}).

Pour résumer :

\begin{enumerate}
\def\labelenumi{\arabic{enumi}.}
\tightlist
\item
  clic sur la référence: copier la clé
\item
  coller ou CTRL+V : colle la clé dans le texte où est positionné le
  curseur (\protect\hyperlink{ref-goody_raison_1979}{Goody 1979}).
\end{enumerate}

Il est également possible d'ajouter une référence ainsi : « Comme le dit
Goody (\protect\hyperlink{ref-goody_raison_1979}{1979}), le geste
\ldots{} »

La clé peut aussi être accompagnée de précision comme ici
(\protect\hyperlink{ref-goody_raison_1979}{Goody 1979, pp.115}).

Les références citées se retrouveront ensuite à la fin du texte dans la
section \texttt{\#\#\ Bibliographie}

\hypertarget{les-citations}{%
\subsubsection{Les citations}\label{les-citations}}

Une citation dans le corps du texte est indiquée par guillemets « Stat
rosa pristina nomine, nomina nuda tenemus ». Une citation plus longue
peut être indiquée ainsi :

\begin{quote}
Stat rosa pristina nomine, nomina nuda tenemus.

la citation se poursuit avec un second paragraphe.
\end{quote}

\hypertarget{versions-et-export}{%
\subsection{Versions et export}\label{versions-et-export}}

\hypertarget{muxe9tadonnuxe9es}{%
\subsubsection{Métadonnées}\label{muxe9tadonnuxe9es}}

Les métadonnées de l'article s'éditent dans le menu en haut à droite.
Vous pouvez y indiquer le titre, sous-titre, le nom de l'auteur et son
identifiant de l'Orcid\footnote{L'identifiant Orcid permettra de
  récupérer automatiquement l'affiliation et la biographie de l'auteur.},
le résumé et les mot-clés de l'article. Pour les éditeurs de revue, une
série plus complète de métadonnées est également disponible.

\hypertarget{preview-et-annotation}{%
\subsubsection{Preview et annotation}\label{preview-et-annotation}}

Chaque version de votre texte peut être prévisualisée et annotée avec
l'outil d'annotation Hypothes.is. Pour accéder à la preview, cliquez sur
le bouton \textbf{Preview}. Pour accéder à la version html annotable,
cliquez sur le bouton \textbf{Anotate}. Vous pouvez alors partager ces
urls de preview et d'annotation. Chaque url est relative à la version du
document.

\hypertarget{export}{%
\subsubsection{Export}\label{export}}

Plusieurs types d'exports sont disponibles :

\begin{itemize}
\tightlist
\item
  XML Erudit : exporte un fichier xml compatible avec le schéma Erudit
\item
  Zip : comprend les trois sources de l'article : yaml (métadonnées),
  bibtex (bibliographie), md (corps de texte)
\item
  \ldots{}
\end{itemize}

\hypertarget{annotations-suxe9mantiques}{%
\subsection{Annotations sémantiques}\label{annotations-suxe9mantiques}}

Il est possible de structurer sémantiquement votre texte avec des
simples balises.

Il y a deux types d'annotation sémantiques:

\begin{enumerate}
\def\labelenumi{\arabic{enumi}.}
\tightlist
\item
  Des annotations qui concernent un ou plusieurs mots dans le même
  paragraphe
\item
  Des annotations qui concernent plusieurs paragraphes
\end{enumerate}

\hypertarget{annotation-dans-un-paragraphe}{%
\subsubsection{Annotation dans un
paragraphe}\label{annotation-dans-un-paragraphe}}

La syntaxe:

{Voici la thèse de l'article}

Produira en HTML:

{Voici la thèse de l'article}

Dans la preview vous pouvez visualiser les classes:

\begin{itemize}
\tightlist
\item
  these
\item
  definition
\item
  exemple
\item
  concept
\item
  epigraphe
\item
  dedicace
\item
  note
\end{itemize}

\hypertarget{annotation-de-plusieurs-paragraphes}{%
\subsubsection{Annotation de plusieurs
paragraphes}\label{annotation-de-plusieurs-paragraphes}}

La syntaxe:

Ici un paragraphe.

Ici un autre paragraphe.

Produira en HTML:

\begin{Shaded}
\begin{Highlighting}[]

\KeywordTok{\textless{}div} \ErrorTok{class}\OtherTok{=}\StringTok{"maclasse"}\KeywordTok{\textgreater{}}
  \KeywordTok{\textless{}p\textgreater{}}\NormalTok{Ici un paragraphe.}\KeywordTok{\textless{}/p\textgreater{}}
  \KeywordTok{\textless{}p\textgreater{}}\NormalTok{Ici un autre paragraphe.}\KeywordTok{\textless{}/p\textgreater{}}
\KeywordTok{\textless{}/div\textgreater{}}
\end{Highlighting}
\end{Shaded}

Pour plus d'informations, consultez la documentation.

\hypertarget{bibliographie}{%
\subsection{Bibliographie}\label{bibliographie}}

\hypertarget{refs}{}
\begin{CSLReferences}{1}{0}
\leavevmode\vadjust pre{\hypertarget{ref-goody_raison_1979}{}}%
Goody, Jack. 1979. \emph{La {Raison} graphique. {La} domestication de la
pensée sauvage.} Le sens commun. Les Editions de Minuit.

\end{CSLReferences}

\end{document}
